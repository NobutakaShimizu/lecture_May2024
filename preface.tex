\chapter*{序文}
一般に「ランダムウォーク」という用語は文脈によって様々である.
例えば物理学や金融の文脈でブラウン運動を離散化したモデルを考える際は
数直線上を等確率で左右どちらかに移動する粒子の軌跡をランダムウォークと呼ぶことがある.
一方でネットワーク解析の文脈ではグラフ上の単純ランダムウォークをランダムウォークと呼ぶこともある.

どの理論でも大体そうだが, 文脈に応じて様々な捉え方があり,それぞれに適した定義がされる.
ある定義が別の定義を特殊ケースとして含んでいるようなこともあるかもしれない.
一方でその特殊性はときに重要な意義を持ち, その分野において重要な役割を果たすことがある.
そしてこの独自性を俯瞰的に見て一般化した理論を整備していくこともまた
    数学の重要なプロセスであろう.

本講義では高次元エクスパンダーと呼ばれる近年の理論計算機科学の多くのブレイクスルーの立役者について解説する.
\emph{理論計算機科学 (theoretical computer science)} とは計算機の能力や限界に迫る分野であり, いわゆる応用数学の一つだが, そのレイヤーは(機械学習や最適化と比べると)非常に低い階層にあるため,
    実は純粋数学の抽象的な概念や道具をある程度の原型を保ったまま応用できる貴重な分野である.
例を挙げればキリがないが, 例えば私がパッと思いつくものを挙げると
\begin{itemize}
    \item 楕円曲線に基づく楕円曲線暗号
    \item 加法的組合せ論に基づく学習器のブースティングやアルゴリズムの設計
    \item 関数解析の道具に基づくランダムウォークの収束性解析
    \item 代数学の理論に基づくエクスパンダーグラフの構成とそれに基づく脱乱択化
    \item 楕円曲線から得られる代数幾何符号
\end{itemize}
がある.
その中でグラフ理論のエクスパンダー性と呼ばれる性質を単体複体に拡張した
    高次元エクスパンダーと呼ばれる概念が近年の理論計算機科学の大きな潮流となっている.

本講義ではまず理論計算機科学においてスタンダードなランダムウォークの定義に基づいてグラフや単体複体上のランダムウォークの理論を構築していく.
私は数学科を出ているわけではないから, 残念ながら高度に抽象的な理論に明るくない.
おそらく本講義で展開される理論の一部はより高度な抽象化がすでに知られていて, その枠組みから簡単に導出できると思われる.
例えば単体複体上のランダムウォークをさらに抽象化することができるかもしれない.
こういった抽象化や理論の整備は純粋数学の人間の方が間違いなく得意であろうことから,
もしも講義でそのようなアイディアが思いついたら
メールでも話しかけるでも何でも良いので気軽に私にご指摘いただけると幸いである
(仮に誤った指摘であったとしても, その視点は私自身の勉強になるので非常に有難い).
本講義が, 日本国内における純粋数学と理論計算機科学の間のつながりをより強固にするきっかけの一つになれば幸甚である.