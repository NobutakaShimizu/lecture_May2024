\section{グラフ上のランダムウォーク}
\subsection{グラフ}
まず基礎的な概念の定義を与える.
本講義では有限単純無向グラフを単に\emph{グラフ (graph)}と呼ぶ.
すなわち, グラフとは有限集合$V$とその二元部分集合$E\subseteq \binom{V}{2}$の組 $G = (V, E)$ である.
$V$の元を\emph{頂点(vertex)}, $E$の元を\emph{辺(edge)}と呼ぶ.
特に混乱が生じない限りは辺$\{u,v\}$を省略して$uv$と記す.
二頂点$u,v\in V$が$uv\in E$を満たすとき, $u$は$v$に隣接しているという ($v$もまた$u$に隣接している).
頂点$u\in V$に対し$\deg(u) = \abs{\cbra{v \in V \colon uv\in E}}$を\emph{次数 (degree)}と呼ぶ.
全ての頂点の次数が$d$に等しいとき, $G$は\emph{$d$-正則 ($d$-regular)}であるという.
以下で定義される行列$A \in \Real^{V\times V}$を\emph{隣接行列 (adjacency matrix)}という:
\begin{align*}
  A(u,v) = \begin{cases}
    1	& \text{if }uv\in E,\\
    0 & \text{otherwise}.
  \end{cases}
\end{align*}
考えているグラフ$G$が明らかな場合は次数や隣接行列などを$\deg(u),A$などと表す.
また, 考えているグラフ$G$が曖昧であったり特別に指定したい場合は$\deg_G(u),A_G$などと表す.

\subsection{グラフ上の単純ランダムウォーク}
ランダムウォークは高次元エクスパンダーの定義やその解析に不可欠な概念である.
そこで本節ではまず最も基本的なグラフ上の単純ランダムウォークを定義し, その基本的な性質を紹介する.
後に単純ランダムウォークを拡張した一般的なランダムウォークを導入し, その重要なクラスである
可逆なランダムウォークについて説明する.
%
\begin{definition}{単純ランダムウォーク}{SRW}
  グラフ$G=(V,E)$を考える.
  頂点集合$V$上に値をとる確率変数の列$(X_t)_{t=0,1,\dots}$であって,
  任意の$t\ge 0$, 頂点列$(v_0,\dots,v_{t-1})\in V^t$, および$v\in V$に対して
  \[
    \Pr\sbra*{ X_t = v \condition X_0=v_0,\dots,X_{t-1} = v_{t-1} } = \Pr\sbra*{ X_t = v_t \condition X_{t-1} = v_{t-1}} = \frac{1}{\deg(v_{t-1})}
  \]
  を満たすものを$G$上の\emph{単純ランダムウォーク (simple random walk)}という.
  さらに,
  \[
    P(u,v) = \begin{cases}
      \frac{1}{\deg(u)}	& \text{if }uv\in E,\\
      0 & \text{otherwise}
    \end{cases}
  \]
  で定義される行列$P \in [0,1]^{V\times V}$を\emph{遷移確率行列 (transition matrix)}と呼ぶ.
\end{definition}
本稿では連続時間の確率過程は考えないので, $(X_t)_{t=0,1,\dots}$は略して$(X_t)_{t\ge 0}$と表す.
単純ランダムウォークの初期地点$X_0$については何も仮定していない.
例えば$X_0$は決定的な頂点$X_0=u$であったり一様ランダムな頂点であっても単純ランダムウォークと呼ぶ.

初期頂点$X_0$の分布が決まれば各時刻$t$における$X_t$の分布は一意に定まる.
実際, $t\ge 0$に対し$x_t \in [0,1]^{V}$を$X_t$の分布とする
  (すなわち, $x_t(u) = \Pr\sbra*{X_t = u}$).
任意の$t\ge 1$に対し
\begin{align*}
  x_t(v) &= \Pr\sbra*{X_t = v} \\
    &= \sum_{u\in V}\Pr\sbra*{ X_t = v \tand X_{t-1} = u} \\
    &= \sum_{u\in V}\Pr\sbra*{ X_t = v \condition X_{t-1} = u} \Pr\sbra*{X_{t-1} = u}  \\
    &= \sum_{u\in V} P(u,v) x_{t-1}(u)
\end{align*}
という漸化式を得る.
これは$x_{t} = x_{t-1} P$とも表せる (ここで$x_t$は行ベクトルとして扱う) ので
\begin{align}
  x_t = x_0 P^t \label{eq:x_t SRW}
\end{align}
を得る.

  グラフ$G=(V,E)$の各頂点を対角に並べた$V\times V$行列を\emph{次数行列 (degree matrix)}という.
  すなわち, 次数行列$D\in \Real^{V\times V}$は
  \begin{align*}
    D(u,v) = \begin{cases}
      \deg(u)	& \text{if }u=v,\\
      0 & \text{otherwise}.
    \end{cases}
  \end{align*}
  グラフ$G$上の単純ランダムウォークの遷移確率行列$P$は, 次数行列$D$と隣接行列$A$を用いて$P=D^{-1}A$と表せる.

\subsection{単純ランダムウォークの収束性と定常分布}
グラフ$G$上の単純ランダムウォーク$(X_t)_{t\ge 0}$を考え, 時刻$t$における分布を$x_t \in [0,1]^V$の
$t\to \infty$における収束性を議論したい.
まず, 分布間の距離として全変動距離を定義する.
\begin{definition}{全変動距離}{total variation distance}
  有限集合$V$上の二つの分布$\mu,\nu \in[0,1]^V$に対し, \emph{全変動距離 (total variation distance)}を
  \[
    \dtv (\mu ,\nu ) \defeq \frac{1}{2} \sum_{u\in V}\abs{\mu(u) - \nu(u)} = \frac{1}{2} \norm{\mu - \nu}_1
  \]
  で定める.
\end{definition}


\subsection{混交時間とエクスパンダーグラフ}