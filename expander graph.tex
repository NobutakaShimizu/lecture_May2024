\chapter{ランダムウォークの固有値とエクスパンダーグラフ}
遷移確率行列$P \in [0,1]^{V \times V}$に従う$V$上の既約的かつ非周期的なランダムウォーク$(X_t)_{t\ge 0}$を考え,
その一意な定常分布を$\pi$とする.
\Cref{thm:random walk convergence}より収束性が保証されるが, その速さを遷移確率行列の固有値に基づいて評価するのが本チャプターの目標である.
この議論の土台となるのがランダムウォークの可逆性という概念である.
一般の遷移確率行列は対称とは限らないが, 可逆性を仮定することによって
遷移確率行列を対称行列として扱うことができ,
対称行列に対して展開される固有値分解などの理論を同じように
遷移確率行列に対しても適用することができる.
これに基づいて既約性, 非周期性, 可逆性を持つランダムウォークの混交時間の上界を与える.
また, エクスパンダーグラフの概念をそのグラフ上の単純ランダムウォークに基づいて定義し,
その組合せ論的な性質をいくつか紹介する.

\section{ランダムウォークの固有値と可逆性}
遷移確率行列$P \in [0,1]^{n \times n}$の固有値
\footnote{本講義では左固有値, すなわち$Px=\lambda x$を満たす$\lambda\in \Comp$を考える.}
$\lambda_1,\dots,\lambda_n$について考える.
全ての成分が$1$であるベクトル$\allone\in \Real^n$を考えると$P\allone = \allone$であるから
$P$は固有値$1$を持つことがわかる.
また, 以下の結果が知られている (Perron--Frobeniusの定理やGershgorinの定理から従う):
\begin{lemma}{遷移確率行列の固有値}{random walk eigenvalue}
    既約的かつ非周期的なランダムウォークの
    遷移確率行列$P\in [0,1]^{n\times n}$の固有値を$\lambda_1,\dots,\lambda_n$とすると,
    全ての$i\in\{1,\dots,n\}$に対して$\abs{\lambda_i} \le 1$を満たす.
    さらに,固有値$1$の多重度は$1$であり (つまり固有値$1$に対応する固有空間は$\{c\allone \colon c \in \Real\}$となる), 他の固有値は全て絶対値が真に$1$より小さい.
\end{lemma}
%

一方でこれまで見ていたグラフ上のランダムウォークは可逆性と呼ばれる嬉しい性質を持っており,
ある意味で対称行列のように扱うことができる.
\begin{definition}{可逆性}{reversible}
    遷移確率行列$P$を持つ$V$上のランダムウォークは,
    ある$V$上の分布$\pi \in [0,1]^V$が存在して
    \begin{align}
        \forall u,v\in V,\pi(u) P(u,v) = \pi(v) P(v,u) \label{eq:reversible}
    \end{align}
    を満たすとき\emph{可逆 (reversible)}であるという.
\end{definition}
\cref{eq:reversible}で表される条件を\emph{詳細釣り合い条件 (detailed balanced equation)}という.
分布$\pi\in [0,1]^V$を対角成分に並べた対角行列$\Pi\in[0,1]^{V\times V}$を用いると
$(\ref{eq:reversible}) \iff \Pi P = (\Pi P)^{\top}$
となる.

可逆性とは直感的に言うと, 逆再生しても同じ分布のランダムウォークになるという性質である.
ランダムウォーク$(X_t)_{t\ge 0}$であって初期頂点$X_0$が分布$\pi$に従って選ばれたものを考えよう.
適当な時刻$t=T$で打ち切って得られる頂点列$(X_0,\dots,X_T)$に対し, 順序を逆にした系列
$(X_T,\dots,X_0)$はランダムウォークから得られた系列と見做せるだろうか?
仮にこれがある遷移確率行列$P^*\in[0,1]^{V\times V}$に従うランダムウォークであったとしよう.
簡単のため$T=1$とする ($T\ge 2$に関しても同じ議論が適用できる).
初期頂点$X_0$の分布$\pi$が定常分布だとすると, $X_1$の分布も$\pi$である.
また, ランダムウォークの条件から$\Pr[X_1 = v \tand X_0=u] = \pi(u)P(u,v)$である.
従って条件付き確率の定義より
\begin{align}
    P^*(v,u) = \Pr[X_0 = u | X_1 = v] = \frac{\Pr[X_0 = u \tand X_1 = v]}{\Pr[X_1 = v]} = \frac{\pi(u)P(u,v)}{\pi(v)} \label{eq:reversal chain}
\end{align}
が得られる.
もし元のランダムウォークが可逆ならば, \cref{eq:reversible}から$P^*=P$を得る.
すなわち, ランダムウォークの可逆性とはそのランダムウォークが時間反転に関して対称性を持つことを意味する.
なお, \cref{eq:reversal chain}で得られる遷移確率行列に従って生成されるランダムウォークを\emph{時間反転ランダムウォーク (time-reversal random walk)}と呼ぶ.

\paragraph*{例1.単純ランダムウォーク}
連結グラフ$G=(V,E)$上の単純ランダムウォークを考えよう.
\cref{eq:SRW stationary distribution}で与えられる定常分布を$\pi$とすると
任意の二頂点$u,v\in V$に対して
\[
    \pi(u) P(u,v) = \frac{\deg(u)}{2|E|} \cdot \frac{\indicator{\{u,v\}\in E}}{\deg(u)}
    = \frac{\deg(v)}{2|E|} \cdot \frac{\indicator{\{u,v\}\in E}}{\deg(v)}
    = \pi(v) P(v,u)
\]
より, 単純ランダムウォークは可逆である (連結なので全ての頂点に対して$\deg(u)>0$である).
ここで, $\indicator{\dots}$は指示関数である.

\paragraph*{例2.重み付きランダムウォーク}
重みつきグラフ$G=(V,E,W)$上の重み付きランダムウォークを考えよう.
重み行列$W$の成分和を$S=\sum_{u,v\in V} W(u,v)$として
分布$\pi(u) = \frac{\deg_W(u)}{S}$を考えると,
\[
    \pi(u) P(u,v) = \frac{\deg_W(u)}{S} \cdot \frac{W(u,v)}{\deg_W(u)}
    = \frac{\deg_W(v)}{S} \cdot \frac{W(v,u)}{\deg_W(v)}
    = \pi(v) P(v,u)
\]
より, 可逆である.

\paragraph*{例3.有向グラフ上のランダムウォーク}
可逆で\emph{ない}例として次の遷移確率行列で与えられるランダムウォークを考えてみよう:
\begin{align*}
    P = \begin{pmatrix}
            0 & 1 & 0 \\
            0 & 0 & 1 \\
            1 & 0 & 0
        \end{pmatrix}.
\end{align*}
この例は遷移が決定的なのでランダムウォークとしては面白くないが,
次のようにして可逆でないことが確認できる:
頂点集合を$V=\{0,1,2\}$とし, $i\in V$に対して$(i+1)\bmod 3$を省略して$i+1 \in V$と書くと
\begin{align*}
    \pi(i) = \pi(i) P(i, i+1) = \pi(i+1)P(i+1,i) = 0
\end{align*}
より$\pi=0$となってしまい, 分布であることに矛盾.

\begin{exercise}{easy}{prob2}
    可逆なランダムウォークの遷移確率行列を$P$とする.
    \cref{eq:reversible}を満たす分布$\pi$は定常分布であることを示せ.
\end{exercise}

%


\section{定常分布から定まる内積とノルム}
可逆なランダムウォークは遷移確率行列の固有値を考える上で非常に扱いやすいランダムウォークのクラスとなっている.
ランダムウォークの分布は遷移確率$P$を右から掛けて得られるのに対し,
    固有値に基づく議論では左固有値を考えており,
この左右の差異に違和感を覚える読者もいるであろう.
実は可逆なランダムウォークでは遷移確率行列を対称行列のように扱うことができ,
ゆえに左右どちらから作用させようが本質的に同じとなる.
このことを説明するために, $\Real^V$に次の内積を導入する.
\begin{definition}{}{naiseki}
    有限集合$V$上の分布$\pi\in(0,1]^V$に対し,
    $\Real^V$に以下の内積$\piprod{\cdot,\cdot}$を定めた内積空間を$\pispace$で表す:
    \begin{align*}
        \piprod{f,g} \defeq \sum_{u \in V} \pi(u) f(u) g(u)
        = f^\top \Pi g.
    \end{align*}
    ここで$f,g$は列ベクトルとして扱い, $\Pi=\mathrm{diag}(\pi)$はベクトル$\pi$の成分を対角に並べた行列である.
    また, 内積$\piprod{\cdot,\cdot}$が誘導するノルムを$\pinorm{\cdot}$で表す.
    すなわち, $f\in\Real^V$に対して
    \[
        \pinorm{f} \defeq \sqrt{\piprod{f,f}}.
    \]
    %    二つのベクトル$f,g \in \Real^V$が$\piprod{f,g} = 0$であるとき, $f \piorth g$で表す.
\end{definition}
\cref{def:naiseki}で考える分布$\pi$は全ての成分が正であるため,
上記の内積$\piprod{\cdot,\cdot}$はちゃんと実ベクトル空間の内積の公理(対称双線形性, 非退化性, 半正定値性)を満たしており,
確かに$\pispace$は内積空間である.

$\Real^V$上の通常の内積$\abra{\cdot,\cdot}$を考えたとき, 任意の対称行列$M \in \Real^{V\times V}$とベクトル$f,g\in \Real^V$に対して $\abra{f,Ag} = \abra{Af,g}$
が成り立っていたが,
可逆なランダムウォークの遷移確率行列$P$は内積$\piprod{\cdot,\cdot}$に関して同様の性質を持つ.
\begin{lemma}{}{reversible adjoint}
    定常分布$\pi$をもつ可逆なランダムウォークの遷移確率行列$P$は,
    任意の$f,g\in\Real^V$に対して
    \[ \piprod{f,Pg} = \piprod{Pf,g} \]
    を満たす.
\end{lemma}
\begin{proof}
    定常分布$\pi$を対角成分に並べた対角行列$\Pi$を考えると
    \begin{align*}
        \piprod{f,Pg} & = f^\top \Pi P g                                              \\
                      & = f^\top (\Pi P)^\top g &  & \text{$\because$可逆性より$\Pi P$は対称} \\
                      & = (Pf)^\top \Pi g       &  & \text{$\because\Pi^\top = \Pi$}  \\
                      & = \piprod{Pf,g}.
    \end{align*}
\end{proof}
一般に$P$が可逆とは限らない場合,
\cref{eq:reversal chain}で与えられる時間反転ランダムウォークの遷移確率行列$P^*$に対し$\piprod{f,Pg} = \piprod{P^*f,g}$が成り立つ.
この意味で$P^*$は$P$の随伴とみなすことができる.


対称行列に対して展開される固有値分解などの理論は可逆なランダムウォークの遷移確率行列に対しても同様に展開できる.
例えば, 対称行列と同様に可逆なランダムウォークの遷移確率行列は実固有値をもつ.
\begin{lemma}{実固有値性}{reversible real eigenvalue}
    既約的かつ可逆なランダムウォークの遷移確率行列$P$と定常分布$\pi$に対し,
    行列
    \begin{align}
        A \defeq \sqrt{\Pi} P \sqrt{\Pi}^{-1} \label{eq: symmetrized P}
    \end{align}
    を考える.
    $P$と$A$は(多重度も含め)同じ固有値をもち, これらは全て実数である.
\end{lemma}
\begin{proof}
    行列$A$は対称である.
    実際,
    \begin{align*}
        A^\top & = \sqrt{\Pi}^{-1} P^{\top} \sqrt{\Pi}                            &  & \text{$\because$$\sqrt{\Pi},\sqrt{\Pi}^{-1}$は対称} \\
               & = \sqrt{\Pi} \cdot \Pi^{-1} P^{\top} \Pi \cdot \sqrt{\Pi}^{-1}                                                         \\
               & = \sqrt{\Pi} \cdot \Pi^{-1} (\Pi P)^{\top} \cdot \sqrt{\Pi}^{-1}                                                       \\
               & = \sqrt{\Pi} \cdot \Pi^{-1} \Pi P \cdot \sqrt{\Pi}^{-1}          &  & \text{$\because$可逆性より$\Pi P$は対称}                 \\
               & = A.
    \end{align*}
    $A$は対称なので全ての固有値は実数である.

    $A$の固有値$\lambda$に対する固有ベクトルを$x$とし, ベクトル$y\defeq \sqrt{\Pi}^{-1}x$を考える.
    固有ベクトルの式
    \begin{align*}
        A x = (\sqrt{\Pi} P \sqrt{\Pi}^{-1})x =  \lambda x
    \end{align*}
    の両辺に左から$\sqrt{\Pi}^{-1}$を掛けると
    \begin{align*}
        P y = \lambda y
    \end{align*}
    を得る.
    すなわち, $P$と$A$は同じ固有値を持つ.
    特に, $P$の固有値も全て実数である.
\end{proof}
\cref{lem:reversible real eigenvalue}において既約性の仮定は除去できる.
実際, $P$が定める状態遷移を表す有向グラフを強連結成分に分解し,
各成分ごとに\cref{lem:reversible real eigenvalue}を適用すればよい.

%
\begin{theorem}{固有分解}{eigendecomposition}
    既約的かつ可逆なランダムウォークの遷移確率行列を$P$とし, その定常分布を$\pi$とする.
    $|V|=n$とする.
    $P$の固有値を$1=\lambda_1\ge \dots \ge \lambda_n \ge -1$とする.
    空間$\pispace$の正規直交基底$x,\dots,x_n$が存在して任意の$t\ge 1$に対して
    \[ P^t\Pi^{-1} = \sum_{i=1}^n \lambda_i^t x_i x_i^{\top}  \]
    と表せ, さらに各$x_i$は$P$の固有値$\lambda_i$に対応する固有ベクトルとなる.
    
    特に$x_1 = \allone$であり,
    $J \in \Real^{V \times V}$を全成分が$1$の行列とすると
    \[ P^t \Pi^{-1} - J =  \sum_{i=2}^n \lambda_i^t x_i x_i^\top \]
    と表せる.
\end{theorem}
\begin{proof}
    \cref{eq: symmetrized P}で定義された行列$A$は対称なので,
    対称行列に対する固有分解の定理より,
    通常の内積$\abra{\cdot, \cdot}$の意味での$\Real^V$の正規直交基底$y_1,\dots,y_n$が存在して
    \[
        A = \sum_{i=1}^n \lambda_i y_i y_i^\top
    \]
    と表せ, さらに各$y_i$は$A$の固有値$\lambda_i$に対応する固有ベクトルとなる.
    一方で$A^t = \sqrt{\Pi} P^t \sqrt{\Pi}^{-1}$だから,
    \[
        \sqrt{\Pi} P^t \sqrt{\Pi}^{-1} = \sum_{i=1}^n \lambda_i^t y_i y_i^\top.
    \]
    両辺に左右から$\sqrt{\Pi}^{-1}$を一つずつ掛けて
    $x_i = \sqrt{\Pi}^{-1}y_i$とおくと
    \[
        P^t \Pi^{-1} = \sum_{i=1}^n \lambda_i^t x_i x_i^\top
    \]
    を得る.
    ここで
    \[
        \piprod{x_i,x_j} = \abra{\sqrt{\Pi}x_i,\sqrt{\Pi}x_j} = \abra{y_i,y_j} = \indicator{i=j}
    \]
    より, 確かに$(x_i)_{i=1,\dots,n}$は空間$\pispace$の正規直交基底である.
    さらに
    \[
        Px_i = P\sqrt{\Pi}^{-1}y_i = \sqrt{\Pi}^{-1}Ay_i = \lambda_i x_i
    \]
    より確かに$x_i$は$P$の固有値$\lambda_i$に対応する固有ベクトルである.

    特に, $\lambda_1=1$に対応する$A$の固有ベクトルは$y_1 = (\sqrt{\pi(u)})_{u \in V}$なので,
    対応する$P$の固有ベクトルは$x_1 = \allone$となる.
\end{proof}

\begin{corollary}{}{covariance}
    \cref{thm:eigendecomposition}と同じ仮定の下で
    空間$\pispace$上の関数$f\colon V \to \Real$に対し,
    \begin{align}
        &\Epi f \defeq \piprod{f,\allone} = \sum_{u\in V}\pi(u) f(u), \label{eq:mean}\\
        &\Varpi f \defeq \pinorm{f - \E_\pi \allone}^2 = \E_\pi f^2 - (\E_\pi f)^2 \label{eq:variance}
    \end{align}
    とする.
    任意の$f,g\colon V \to \Real$に対し
    \[
        \abs*{\piprod{f,g} - \E_\pi f\cdot \E_\pi g} \le \max\cbra*{\abs{\lambda_2},\abs{\lambda_n}}\sqrt{\Varpi f \cdot \Varpi g}.
    \]
\end{corollary}
%
\begin{exercise}{}{covariance}
    \cref{cor:covariance}を証明せよ.
\end{exercise}
%
%
\begin{comment}
\paragraph*{遷移確率行列を左右どちらから作用させるか?}
\cref{eq:p_t}では遷移確率行列$P$を右から作用させることで次のステップのランダムウォークの分布が得られることがわかる.
では, 遷移確率行列$P$を左から作用させて得られる線形作用素はどのような意味合いを持つだろうか?
ベクトル$f\in \Real^V$を関数$f\colon V \to \Real$とみなす.
遷移確率行列$P$ を$f$に左から作用させて得られる関数 $Pf\colon V\to \Real$ は
\[
    (Pf)(u) = \sum_{v\in V}P(u,v)f(v) = \mathbb{E}_{v\sim P(u,\cdot)}[f(v)]
\]
と表せる. ここで, $\E_{v\sim P(u,\cdot)}[f(v)]$とは$P$の第$u$行が定める$V$上の分布に従ってランダムに選ばれた頂点$v$に対して$f(v)$の期待値を意味する.
従って, 線形作用素としての$P$は$V$上の関数$f$に対し局所的に$P(u,\cdot)$で重みつけして平均化しているとみなせる.

可逆なランダムウォークを考える限りにおいては\cref{prop:reversible adjoint}から本質的にはどちらを考えても変わらない.
ただし, 応用側の都合で混交時間の定義において全変動距離を用いており, 全変動距離は通常の$\ell^1$ノルムを用いているため, 本来考えるべきであろう自然な$\ell^1$ノルム$f \mapsto \sum_{u \in V} \pi(u)\abs{f(u)}$に対して$\pi_{\min} \defeq \min_{u\in V}\pi(u)$倍程度のギャップが生じる.
\end{comment}

\section{ランダムウォークのスペクトルと混交時間}
非自明な固有値が絶対値の意味で小さいときに混交時間が上から抑えられることを示す.
\begin{definition}{}{second eigenvalue}
    サイズ$n$の集合$V$上の可逆なランダムウォークを考え,
    その遷移確率行列$P$の固有値を$1=\lambda_1 \ge \dots \ge \lambda_n\ge -1$に対し,
    $\lambda(P) \defeq \max\cbra{\abs{\lambda_2},\abs{\lambda_n}}$とする.
    特に, $\gamma\defeq 1-\lambda(P)$を\emph{スペクトルギャップ (spectral gap)}と呼ぶ.
\end{definition}
既約性と非周期性を仮定するとスペクトルギャップは正となる.
スペクトルギャップが大きいほど混交時間が小さくなる.
\begin{lemma}{混交時間とスペクトルギャップ}{mixing time and spectral gap}
    集合$V$上の既約, 非周期的, 可逆なランダムウォークを考え,
    そのスペクトルギャップを$\gamma$とする.
    定常分布$\pi$に対し$\pimin = \min_{u\in V} \pi(u)$とすると,
    任意の初期分布に対して
    \[ \tmix(\varepsilon) \le\frac{\log\rbra*{\frac{1}{2\pimin\varepsilon}}}{\log(1/\lambda)}. \]
    特に, スペクトルギャップが$\gamma>0$のとき,
    $\tmix(\varepsilon) \le \frac{1}{\gamma}\log\rbra*{\frac{1}{2\pimin\varepsilon}}$.
\end{lemma}
%
\begin{proof}
    \cref{thm:eigendecomposition}の正規直交基底を$x_1,\dots,x_n$とする.
    ピタゴラスの定理より任意のベクトル$f \in \Real^V$は
    \[ \pinorm{f}^2 = \sum_{i=1}^n \piprod{f,x_i}^2 \]
    を満たす.
    特に, 頂点$u$を固定し$f$としてディラック測度$f=\delta_u$とすると
    \begin{align*}
        \pi(u) & = \pinorm{\delta_u}^2                                                           \\
               & = \sum_{i=1}^n \piprod{\delta_u,x_i}^2                                          \\
               & = \sum_{i=1}^n \pi(u)^2x_i(u)^2                                                 \\
               & = \pi(u)^2 + \pi(u)^2\sum_{i=2}^n x_i(u)^2 &  & \text{($\because x_1=\allone$)}
    \end{align*}
    を得る.
    特に, $\sum_{i=2}^n x_i(u)^2 = \frac{1}{\pi(u)} - 1 \le \frac{1}{\pi(u)}$である.

    ここで, \cref{thm:eigendecomposition}より, $P^t\Pi^{-1} - J$の第$(u,v)$成分に着目すると
    \begin{align*}
        \abs*{\frac{P^t(u,v)}{\pi(v)} - 1} & \le \sum_{i=2}^n \abs{\lambda_i}^t \abs{x_i(u)x_i(v)}                                                                           \\
                                           & \le \lambda^t \sum_{i=2}^n  \sqrt{\sum_{i=2}^n x_i(u)^2} \sqrt{\sum_{i=2}^n x_i(v)^2} &  & \text{$\because$Cauchy--Schwarzの不等式} \\
                                           & \le \frac{\lambda^t}{\sqrt{\pi(u)\pi(v)}}                                                                                       \\
                                           & \le \frac{\lambda^t}{\pimin}
    \end{align*}
    を得る.
    特に, 任意の頂点$u\in V$に対して
    \[
        \dtv\rbra*{P^t(u,\cdot),\pi} = \frac{1}{2}\sum_{v\in V} \abs*{P^t(u,v) - \pi(v)} \le \frac{\lambda^t}{2\pimin}
    \]
    なので, 任意の初期分布に対して混交時間は
    \[
        \tmix(\varepsilon) \le \inf\cbra*{t\ge 0\colon \frac{\lambda^t}{2\pimin} \le \varepsilon} \le \frac{\log\rbra*{\frac{1}{2\pimin\varepsilon}}}{\log(1/\lambda(P))} \le \frac{1}{\gamma}\log\rbra*{\frac{1}{2\pimin\varepsilon}}.
    \]
    最後の不等式では$\forall x\in \Real,x\le \e^{x-1}$を用いた.
\end{proof}

\section{エクスパンダーグラフ}
グラフ$G=(V,E)$は, 単純ランダムウォークの非自明な第二固有値$\lambda(P)$が小さいときにエクスパンダーであるという.
多くの文脈では通常, 正則グラフに対してのみエクスパンダー性が定義されるが
本講義では一般のグラフに対してエクスパンダー性を定義する.
\begin{definition}{エクスパンダー}{expander}
    グラフ $G=(V,E)$上の遷移確率行列$P$が$\lambda(P) \le \lambda$を満たすときグラフ$G$は\emph{$\lambda$-エクスパンダー ($\lambda$-expander)}という.
    また, $P$の第二固有値が$\lambda_2 \le \lambda$を満たすとき, グラフ$G$は\emph{片側$\lambda$-エクスパンダー (one-sided $\lambda$-expander)}という.
\end{definition}
本講義ではエクスパンダー性を持つ単体複体も取り扱うため,
エクスパンダー性を持つグラフのことを\emph{エクスパンダーグラフ}と呼んで区別する.

要するにグラフのエクスパンダー性とは単純ランダムウォークの混交時間が小さいという性質を意味する.
二部グラフは周期的であり特に最小固有値が$\lambda_{|V|}=-1$となるためこの意味ではエクスパンダーグラフになりえないが,
片側エクスパンダーであるならば
遅延単純ランダムウォークの混交時間は小さくなる.

ランダムウォークの混交時間が小さいとはランダムウォークが「すぐに混ざり合う」ことを意味する.
この「すぐに混ざり合う」性質から, ランダムウォーク$(X_t)_{t\ge 0}$が時刻$t$までに訪れた頂点の集合を$U_t = \cbra*{X_0,\dots,X_t}$とすると, $\abs{U_t}$はすぐに拡大(expand)していく.

\paragraph*{例1 完全グラフ.}
\paragraph*{例2 閉路グラフ.}
\paragraph*{例3 Petersenグラフ.}

\subsection{エクスパンダーの擬似ランダム性(*)}
この節は高次元エクスパンダーの本筋から少し外れるが,
エクスパンダーグラフの重要であることの理由の一つとしてその擬似ランダム性について概説する.

加法的組合せ論や計算量理論では\emph{擬似ランダム性 (pseudorandomness)}と呼ばれる概念が非常に重要な役割を果たしている.
\begin{definition}{分布の擬似ランダム性}{pseudorandomness}
    有限集合$\Omega$上のある分布$\mu$と関数族$\mathcal{F}=\{f\colon \Omega \to \binset\}$を考える.
    分布$\mu$は,
    任意の$f\in \mathcal{F}$に対して
    \[ \abs*{\E_{x\sim \mu} [f(x)] - \E_{y\sim U_{\Omega}}[f(y)]} \le \varepsilon\]
    を満たすとき, \emph{$\mathcal{F}$に対して$\varepsilon$-擬似ランダムである}という (ここで, $y\sim U_\Omega$とは$\Omega$上一様ランダムに$y$が選ばれたことを意味する).
\end{definition}
直感的には, 分布が擬似ランダムであるとは, その分布が任意の$f\in \mathcal{F}$を使っても一様分布と\emph{識別できない (indistinguishable)}ことを意味する.
例えば全変動距離に関する\cref{prop:dtv}では, $\mathcal{F}$を$V$上の二値関数全体 (すなわち任意の$V$の部分集合)の族としたときの識別不可能性のパラメータ$\varepsilon$が全変動距離で与えられることを意味する.
すなわち$\mu$は常に$\dtv(\mu,U_{\Omega}))$-擬似ランダムである.
関数クラス$\mathcal{F}$をより制限したときにパラメータ$\varepsilon$がどこまで小さくなるかに興味がある.

組合せ論では$\mathcal{F}$としてある特殊な関数クラスを仮定することによって\emph{組合せ論的擬似ランダム性}を定義する.
例えばグラフ理論や加法的組合せ論のコーナーストーンの一つと呼ばれるSzemerédiの正則化補題と呼ばれる結果は, 非常に大雑把に言えば
任意の密なグラフが定数個の擬似ランダムな二部グラフと疎な部分に分解できることを主張する定理である.
組合せ論的擬似ランダムネスの概念は特に加法的組合せ論において非常な協力な道具となっており,
Green--Taoの定理の証明においても重要な役割を果たしている
(驚くべきことに, 識別不可能性の枠組みでGreen--Taoの定理の証明を理解してそれを学習理論におけるブースティングの証明に応用するという研究もなされている!).

計算量理論では$\mathcal{F}$を「効率的なアルゴリズムの全体」や「素子数の少ない論理回路の全体」とすることで\emph{計算量的擬似ランダム性}を定義できる.
任意の効率的なアルゴリズムに対して一様ランダムな文字列と識別できないということは, その分布に従って生成されたメッセージを盗み見てもそこから得られる情報が何もない (ランダムな文字列を見てるのと同じ) であることから, 計算量的擬似ランダム性は暗号の計算量的安全性の定義の根幹をなすことがわかる.

エクスパンダーグラフの組合せ論的擬似ランダム性を説明する.
正則$\lambda$-エクスパンダー$G=(V,E)$を考える.
集合$\Omega=V\times V$上の分布$\mu = \mu_G$として
一様ランダムな辺$\{u,v\}\in E$を選び, $(u,v)$もしくは$(v,u)$どちらかを等確率で選んだ時の頂点対の分布とする.
すなわち,
\begin{align}
    \Pr_{(u,v)\sim \mu}\sbra*{ (u,v) = (s,t)} = \frac{\indicator{\{s,t\}\in E}}{2\abs{E}} = \frac{\indicator{\{s,t\}\in E}}{nd} \label{eq:expander mu}
\end{align}
とする.
関数族$\mathcal{F}$を
\begin{align}
    \mathcal{F} = \cbra*{ f_{S,T} \colon (s,t) \mapsto \indicator{s\in S,t\in T} \colon S,T\subseteq V}  \label{eq:expander F}
\end{align}
で定める.
\begin{lemma}{エクスパンダー混交補題}{expander mixing lemma}
    グラフ$G$が$n$頂点$d$-正則$\lambda$-エクスパンダーであるとき, \cref{eq:expander mu}で定義された分布$\mu$は\cref{eq:expander F}で定義された関数族$\mathcal{F}$に関して$\varepsilon$-擬似ランダムである.
    すなわち, 任意の頂点部分集合$S,T\subseteq V$に対して,
    $e(S,T) = \sum_{s\in S,t\in T} \indicator{\{s,t\} \in E}$を$S,T$間の辺の本数($S\cap T$内の辺は2回数える)とすると,
    \[
        \abs*{e(S,T) - \frac{d}{n}|S||T|} \le d\lambda\sqrt{|S||T|\rbra*{1-\frac{|S|}{n}}\rbra*{1-\frac{|T|}{n}}}.
    \]
\end{lemma}
\begin{proof}
    \Cref{cor:covariance}を適用する.

\end{proof}

\subsection{エクスパンダー性の限界とラマヌジャングラフ}
あとの節(\cref{sec:expander graph application})で説明するが,
応用上では正則なエクスパンダーグラフであってできるだけ辺の少ないものが重要である.

\section{エクスパンダーグラフの応用} \label{sec:expander graph application}
グラフのエクスパンダー性は組合せ論的な興味だけでなく,
理論計算機科学において多くの定理の証明の道具として非常に重要な役割を果たしている.
ここではその一端を軽く紹介する.
%
\subsection{脱乱択化}
Albert Einsteinは「量子は確率的に振る舞う」とする量子力学の枠組みに対して懐疑的であり,
1926年にMax Bornに宛てた手紙において
\begin{quotation}
    Der Alte würfelt nicht. (神はサイコロを振らない)
\end{quotation}
と述べている.
では, アルゴリズムの神はサイコロを振るだろうか?
より具体的には, 乱択は計算能力を真に向上させるだろうか?
この哲学的な問いは90年代から今もなお計算量理論において深く研究されており,
その中心的なリーダーの一人であるAvi Wigdersonは2021年にAbel賞, 2023年にTuring賞を受賞している.

ここではエクスパンダーグラフを使って「少ないサイコロで多くのサイコロの出目をhitting性の意味で模倣できる」
という結果を紹介する.
$R_1,\dots,R_{t} \in \binset^n$を独立な確率変数で各$R_i$を$\binset^n$上一様ランダムな文字列とする.
集合$S\subseteq \binset^n$を一つ固定し, $\abs{S} \ge \delta 2^n$とする.
直感的にはある$n$ビットの乱数を用いて確率$\delta$で成功する乱択アルゴリズム$\mathcal{A}$を考え,
    そのアルゴリズムが成功するランダムシードの集合を$S$としている.
アルゴリズム$\mathcal{A}$を一度実行するだけでは成功確率は$\delta$であるが,
独立に$t$回走らせるとその確率を増加させることができ, $t$回の試行の中で少なくとも一度は成功する確率は
\[
    \Pr[\exists i\in\cbra{1,\dots,t}, R_i \in S] = 1 - (1-\delta)^t \ge 1-\e^{-\delta t}
\]
である.
従って試行回数$t$を増やすと$t$に関して指数的に成功確率が上昇する.
このとき, 全体で用いたランダムビットの長さは$nt$ビットである.

エクスパンダーグラフ上のランダムウォークを考えると
成功確率に少しのロスが生じるものの, ランダムビットの長さを$n+t\log d$ビットに減らすことができる ($d\in\Nat$はパラメータで, 大きくするほど成功確率のロスが小さくできる).
グラフ$G=(\binset^n,E)$を$d$-正則$\lambda$-エクスパンダーグラフとする (頂点集合が$V=\binset^n$であることに注意).

\subsection{誤り訂正符号}

\subsection{PCP定理}

\subsection{Goldreichの擬似乱数生成器}

\subsection{エクスパンダーハッシュ}