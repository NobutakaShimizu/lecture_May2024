\section{ランダムウォークの固有値とエクスパンダーグラフ}
遷移確率行列$P \in [0,1]^{V \times V}$に従う$V$上の既約的かつ非周期的なランダムウォーク$(X_t)_{t\ge 0}$を考え,
その一意な定常分布を$\pi$とする.
\Cref{thm:random walk convergence}より収束性が保証されるが, その速さを評価したい.

\subsection{ランダムウォークの固有値と可逆性}
遷移確率行列$P \in [0,1]^{n \times n}$の固有値$\lambda_1,\dots,\lambda_n$について考える.
全ての成分が$1$であるベクトル$\allone\in \Real^n$を考えると$P\allone = \allone$であるから
$P$は固有値$1$を持つことがわかる.
また, 以下の結果が知られている (Perron--Frobeniusの定理やGershgorinの定理から従う):
\begin{proposition}{遷移確率行列の固有値}{}
    既約的かつ非周期的なランダムウォークの
    遷移確率行列$P\in [0,1]^{n\times n}$の固有値を$\lambda_1,\dots,\lambda_n$とすると,
    全ての$i\in\{1,\dots,n\}$に対して$\abs{\lambda_i} \le 1$を満たす.
    さらに,固有値$1$の多重度は$1$であり (つまり固有値$1$に対応する固有空間は$\{c\allone \colon c \in \Real\}$となる), 他の固有値は全て絶対値が真に$1$より小さい.
\end{proposition}
%

一方でこれまで見ていたグラフ上のランダムウォークは可逆性と呼ばれる嬉しい性質を持っており,
ある意味で対称行列のように扱うことができる.
\begin{definition}{可逆性}{reversible}
    遷移確率行列$P$を持つ$V$上のランダムウォークは,
    ある$V$上の分布$\pi \in [0,1]^V$が存在して全ての$u,v\in V$に対し
    \begin{align}
        \pi(u) P(u,v) = \pi(v) P(v,u) \label{eq:reversible}
    \end{align}
    を満たすとき\emph{可逆 (reversible)}であるという.
\end{definition}
\cref{eq:reversible}で表される条件を\emph{詳細釣り合い条件 (detailed balanced equation)}という.

可逆性とは直感的に言うと, 逆再生しても同じ分布のランダムウォークになるという性質である.
ランダムウォーク$(X_t)_{t\ge 0}$であって初期頂点$X_0$が分布$\pi$に従って選ばれたものを考えよう.
適当な時刻$t=T$で打ち切って得られる頂点列$(X_0,\dots,X_T)$に対し, 順序を逆にした系列
$(X_T,\dots,X_0)$はランダムウォークから得られた系列と見做せるだろうか?
仮にこれがある遷移確率行列$P^*\in[0,1]^{V\times V}$に従うランダムウォークであったとしよう.
簡単のため$T=1$とする ($T\ge 2$に関しても同じ議論が適用できる).
初期頂点$X_0$の分布$\pi$が定常分布だとすると, $X_1$の分布も$\pi$である.
また, ランダムウォークの条件から$\Pr[X_1 = v \tand X_0=u] = \pi(u)P(u,v)$である.
従って条件付き確率の定義より
\[
    P^*(v,u) = \Pr[X_0 = u | X_1 = v] = \frac{\Pr[X_0 = u \tand X_1 = v]}{\Pr[X_1 = v]} = \frac{\pi(u)P(u,v)}{\pi(v)}
\]
が得られる.
もし元のランダムウォークが可逆ならば, \cref{eq:reversible}から$P^*=P$を得る.
すなわち, ランダムウォークの可逆性とはそのランダムウォークを逆再生しても自身と全く同じ分布に従う頂点系列が得られるという性質を意味する.
\todo[inline]{定常分布との関係}
%
\subsection{定常分布が誘導する内積}
可逆なランダムウォークは遷移確率行列の固有値を考える上で非常に扱いやすいランダムウォークのクラスとなっている.
このことを説明するために, 