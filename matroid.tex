\chapter{マトロイド}
マトロイド(matroid)は「行列(matrix)のようなもの(-oid)」という名を冠するが,
線形代数における線型独立性をグラフの全域木などに拡張した概念である.

\section{定義}
\begin{definition}{マトロイド}{matroid}
    次の性質を持つ単体複体$(V,\F)$を\emph{マトロイド (matroid)}という:
    任意の$\sigma,\tau \in \F$に対し, $\abs{\sigma} < \abs{\tau}$ならば,
    ある$ u \in \tau \setminus \sigma$が存在して
    $\sigma \cup \cbra{u} \in \F$.
\end{definition}

\subsection{例1. グラフ的マトロイド}
\subsection{例2. 線形マトロイド}
\subsection{例3. 分割マトロイド}
\section{モチベーション}
\subsection{組合せ最適化}
\subsection{組合せ論}
\section{基の数え上げ}
\section{Anari, Liu, Gharan, Vinzantの定理}
