\chapter{マトロイド} \label{chap:matroid}
マトロイド(matroid)は「行列(matrix)のようなもの(-oid)」という名を冠するが,
線形代数における線型独立性をグラフの全域木などに拡張した概念である.

\section{定義}
\begin{definition}{マトロイド}{matroid}
    次の性質を持つ単体複体$(V,\F)$を\emph{マトロイド (matroid)}という:
    任意の$\sigma,\tau \in \F$に対し, $\abs{\sigma} < \abs{\tau}$ならば,
    ある$ u \in \tau \setminus \sigma$が存在して
    $\sigma \cup \cbra{u} \in \F$.
\end{definition}

\subsection{例1. グラフ的マトロイド}
\subsection{例2. 線形マトロイド}
\subsection{例3. 分割マトロイド}
\section{モチベーション}
\subsection{組合せ最適化}
\subsection{組合せ論}
\section{基の数え上げ}
\section{Anari, Liu, Gharan, Vinzantの定理}
\begin{lemma}{}{ALGV}
    マトロイド$(V,\F)$は局所$0$-エクスパンダーである.
\end{lemma}
\section{その他の応用}
理論計算機科学における(マトロイド以外への)高次元エクスパンダーの応用を簡単にまとめる.
多くの応用は本質的には高次元エクスパンダーがもつ局所大域原理に基づいている.
