\chapter{おわりに}
本講義ではまず, ランダムウォークの混交時間をスペクトルを用いて抑える方法を証明した.
次に, 単純ランダムウォークが高速に混交するグラフとしてエクスパンダーグラフを定義し,
エクスパンダー性の限界や理論計算機科学における応用を紹介した.
そして単体複体上のいくつかの種類のランダムウォークを定義し,
さらにこれに基づいてエクスパンダー性を定義した.
また, 単体複体のエクスパンダー性には局所的なものと大域的なものが定義でき,
局所的なエクスパンダー性から大域的なエクスパンダー性が導けることを示した (局所大域原理).
最後に応用としてトリクルダウン定理を用いてマトロイドのエクスパンダー性を議論し,
長年の未解決問題であったMihail--Vazirani予想の証明を与えた.

最後に, 高次元エクスパンダーの近年の進展について簡単に紹介する.
%
\section{左右ケイリー複体に基づく誤り訂正符号の構成}
$\Code\subseteq \mathbb{F}_2^n$を符号といい, 特に$\Code$が線形部分空間ならば線形符号という (定義については\cref{sec:error correcting code}参照).
線形符号$\Code$の次元を\emph{レート}と呼び,
相異なる二要素間の最小ハミング距離$\min_{x\neq y\in \Code}\dist(x,y)$を$\Code$の\emph{距離}という.
レートと距離がどちらも大きい符号が望ましいとされるが, これらのパラメータにはトレードオフがある.
理論計算機科学ではこれら二つの組合せ論的性質に加えて, 与えられた$w\in\mathbb{F}_2^n$が符号$\Code$に属するかどうかを効率的に判定できるという計算機科学的な性質(局所検査性; \citet{GS06})も重要である.
これら三つの性質を同時に
達成する符号の存在性はSpielmanの博士論文(1996)で
提起されて以来30年近く未解決であり, そのような符号は計算量理論において効率的な確率的検証可能証明(PCP)の構成などに応用される.
近年, \citet{DELLM22}はこの未解決問題を肯定的に解決した.
彼女らは, ケイリーグラフの概念を立方複体に自然に拡張した\emph{左右ケイリー立方複体 (left-right Cayley complex)}を導入し, それに基づく新たな符号を提案した.
この符号は\cite{SS96}によるエクスパンダーグラフに基づく符号(\cref{def:Cayley expander code})の自然な拡張である.

本講義で扱った高次元エクスパンダーはエクスパンダー性をもつ\emph{単体}複体であったが, \cite{DELLM22}で考えられている
左右ケイリー立方複体はある種のエクスパンダー性をもつ\emph{立方}複体である.
\citet{DELLM22}は, 立方複体の四角形が織り成す構造を\emph{テンソル符号}と呼ばれる二つの符号を合成する手法と巧妙に組み合わせることによって, 理論計算機科学の長年の未解決問題を解決することができたのである.
なお, 独立同時期に\citet{LH22,PK22}も同様の手法を用いてこの未解決問題を解決している.
\cite{LH22,PK22}に述べられているように, この手法は\emph{量子LDPC符号 (quantum LDPC code)}の構成にも応用されている \cite{LZ22}.